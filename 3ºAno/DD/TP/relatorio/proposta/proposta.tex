\documentclass{article}
\usepackage{graphicx}
\usepackage[margin=2.5cm]{geometry}
\usepackage{listingsutf8}
\usepackage[portuguese]{babel}

\graphicspath{{./images/}}

\begin{document}
	\begin{titlepage}
    	\begin{center}
    		\includegraphics[width=0.6\textwidth]{logo-isec}
    		
    		\vspace*{\fill}
    		
    		\Huge
    		\textbf{Proposta de trabalho}
    		
    		\huge
    		Disponibilidade e Desempenho
    		
    		\vspace{0.5cm}
    		\LARGE
    		2021 - 2022
    		
    		\vspace{1.5cm}
    		
    		\textbf{Bruno Teixeira}\\
                a2019100036@isec.pt
    		
    		\vfill
    		\vspace*{\fill}
    		
    		\normalsize
    		Licenciatura de Engenharia Informática \\
    		25 de outubro de 2021
    	\end{center}
    \end{titlepage}


\section{Identificação do aluno}      
\textbf{Nome} : Bruno Alexandre Ferreira Pinto Teixeira \\
\textbf{Número de aluno} : 2019100036 \\
\textbf{Email} : a2019100036@isec.pt

\section{Identificação da tecnologia em causa}
\paragraph{}
Balanceamento de carga em servidores web com HAProxy e Keepalived.

\section{Breve contextualização da tecnologia}
\paragraph{}
O HAProxy é um serviço que garante um balanceamento de carga para distribuir o acesso externo por vários servidores. Ao adicionar o Keepalived é possível criar redundância na arquitetura pois aqui são usados pelo menos dois balanceamentos de carga fazendo com que, se um estiver inoperacional o outro entra em ação.

\section{Descrição clara das perguntas a que pretende responder durante o seu estudo}
\begin{itemize}
  \item O que são Proxies?
  \item Quão importante é distribuir pedidos por vários servidores? 
  \item Tipos de balanceamento de carga
  \item Algoritmos usados em balanceamentos de carga
  \item E se o único servidor de HAProxy estiver inoperacional?
\end{itemize}

\section{Descrição clara dos cenários que pretende operacionalizar durante o estudo para encontrar as respostas identificadas}
\paragraph{}Durante o estudo pretendo criar uma aplicação web simples que será usada em dois servidores distintos. Antes do cliente se conectar à aplicação web, o mesmo irá passar pelo HAProxy que será responsável por distribuir o acesso à aplicação. \\
Em conjunto com o HAProxy será usado o Keepalive para obter uma maior redundância no balanceamento de carga da arquitetura final.


\section{Identificação dos recursos necessários para realização prática dos cenários descritos}
\paragraph{}
Para a realização prática do trabalho irei utilizar um servidor ESXi com algumas máquinas virtuais (servidores web e HAProxies) sendo que a aplicação web será apenas acedida na LAN do ESXi.

\end{document}
